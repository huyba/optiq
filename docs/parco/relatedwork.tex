\section{Related work}
\label{sec:relatedwork}

Bandwidth optimization has been studied in great details in the literature. Many works consider a particular system's interconnection information and application's communication patterns to optimize throughput. Essentially, these two characteristics can be used to map an application's processes to specific processors so that interprocessor communication can take advantage of that network.

In an MPI-enabled environment, bandwidth optimization can be done via MPI processes mapping. In \cite{Bhatele:mapping}, the authors provide a tool for performing a wide variety of mappings for structured communication patterns. The tool provided mappings that were able to increase bandwidth and reduce latency and congestion. The tool did not take unstructured communication patterns into account, thus, did not realize that multiple paths could be made available for data movement.

Multiple path data movement was realized in the work of Khanna et al. \cite{Proxies:Gaurav} by using intermediate nodes when an explicit path setup was not provided. This work focused on wide-area networks (WANs) where the exact network topology is hidden from users. Accordingly, shared links are identified through real experimental network throughputs. Our work is applied to the interconnection network and the I/O subsystem of supercomputers where the network topology and associated routing policies are known a priori and the size of the network is much larger than WANs. Our ideas come from the observation that compute nodes in the BG/Q system have 10 links for communication but usually only a single path is used for transferring data between nodes or between nodes and I/O nodes.

Kumar and Faraj \cite{Kumar:Allreduce} proposed using multiple incoming and outgoing links per node for communication on the BG/Q. The work was focused on MPI Allreduce collective communication while our work targets sparse data movement among a subset of nodes.

Adaptive routing for network balancing has also been studied in \cite{Valiant:Routing,singh2003:goal}. In addition, there are works on adaptive routing for current supercomputers such as the BG/Q \cite{Chen:BGQ} and the Cray Cascade \cite{garcia2013:CrayDragonfly}. However, these works are for low-level networking, where any packet can be routed to any path. In contrast, our study leverages underlying routing policies to implement multipath data movement in the user space, where we have more detailed knowledge about the data flows and patterns.

I/O forwarding and staging is routinely used for improving I/O performance to storage. A scalable I/O forwarding framework for high-performance computing systems is presented in \cite{Ali:IOForwarding}. The authors in \cite{Vishwanath:IOForwarding,Vishwanath:GLEAN}  proposed an augmentation for I/O forwarding and asynchronous data staging for BG/P and /Q systems. However, those studies assumed that the data is dense and uniformly distributed. Our work in this paper extends on our previous work \cite{Vishwanath:GLEAN} to deal with sparse data movement patterns.