\section{Conclusions and future works}
\label{sec:conclusions}

In this paper, we present a scalable mechanism for improving sparse data movement, which often takes place in scientific applications such as  in-situ analysis and multiphysics applications, by utilizing underlying resources. Our approaches introduce intermediate nodes to increase data transfer throughput. Our solutions work for both sparse data movement between data coupling groups of nodes and sparse data movement between compute nodes and I/O nodes. We demonstrate the efficacy of our solutions through microbenchmarks and application benchmarks showing up to $2\times$ data movement throughput improvement. Our work shows that network topology aware data movement utilizing all available network resources helps improve the performance of data-intensive applications.
In the future, we plan to employ pipeline technique in which data will be split into small messages to be transferred. Thus, we will need only 2 proxies at least to get benefit from proxies-based method. We will come up with an analytical model for the achievable throughput and explore graph models for data movement in different network topologies and with different shapes of partitions given for physics modules.

%more location of multiphysics coupling nodes, staging...
%We need to varies the size of source and destination regions and see how the performance varies.

%We also need to show that the formula for number of m, n, total number of nodes and with actual performance results and to show that at certain numbers, we do not benefit from using proxies.
