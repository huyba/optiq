%% 
%% Copyright 2007, 2008, 2009 Elsevier Ltd
%% 
%% This file is part of the 'Elsarticle Bundle'.
%% ---------------------------------------------
%% 
%% It may be distributed under the conditions of the LaTeX Project Public
%% License, either version 1.2 of this license or (at your option) any
%% later version.  The latest version of this license is in
%%    http://www.latex-project.org/lppl.txt
%% and version 1.2 or later is part of all distributions of LaTeX
%% version 1999/12/01 or later.
%% 
%% The list of all files belonging to the 'Elsarticle Bundle' is
%% given in the file `manifest.txt'.
%% 

%% Template article for Elsevier's document class `elsarticle'
%% with numbered style bibliographic references
%% SP 2008/03/01

%\documentclass[preprint,12pt]{elsarticle}

%% Use the option review to obtain double line spacing
%% \documentclass[authoryear,preprint,review,12pt]{elsarticle}

%% Use the options 1p,twocolumn; 3p; 3p,twocolumn; 5p; or 5p,twocolumn
%% for a journal layout:
%% \documentclass[final,1p,times]{elsarticle}
%% \documentclass[final,1p,times,twocolumn]{elsarticle}
%% \documentclass[final,3p,times]{elsarticle}
%% \documentclass[final,3p,times,twocolumn]{elsarticle}
 \documentclass[final,5p,times]{elsarticle}
%% \documentclass[final,5p,times,twocolumn]{elsarticle}

%% For including figures, graphicx.sty has been loaded in
%% elsarticle.cls. If you prefer to use the old commands
%% please give \usepackage{epsfig}

%% The amssymb package provides various useful mathematical symbols
\usepackage{amssymb}
%% The amsthm package provides extended theorem environments
%% \usepackage{amsthm}

%% The lineno packages adds line numbers. Start line numbering with
%% \begin{linenumbers}, end it with \end{linenumbers}. Or switch it on
%% for the whole article with \linenumbers.
%% \usepackage{lineno}

\usepackage{algorithm}
\usepackage{algorithmic}

\journal{Parallel Computing}

\begin{document}

\begin{frontmatter}

%% Title, authors and addresses

%% use the tnoteref command within \title for footnotes;
%% use the tnotetext command for theassociated footnote;
%% use the fnref command within \author or \address for footnotes;
%% use the fntext command for theassociated footnote;
%% use the corref command within \author for corresponding author footnotes;
%% use the cortext command for theassociated footnote;
%% use the ead command for the email address,
%% and the form \ead[url] for the home page:
%% \title{Title\tnoteref{label1}}
%% \tnotetext[label1]{}
%% \author{Name\corref{cor1}\fnref{label2}}
%% \ead{email address}
%% \ead[url]{home page}
%% \fntext[label2]{}
%% \cortext[cor1]{}
%% \address{Address\fnref{label3}}
%% \fntext[label3]{}

\title{Improving Data Movement Performance for Sparse Data Patterns on the Blue Gene/Q Supercomputer}

%% use optional labels to link authors explicitly to addresses:
%% \author[label1,label2]{}
%% \address[label1]{}
%% \address[label2]{}

\author[evl]{Huy Bui}
\author[mcs]{Eun-Sung Jung}
\author[mcs]{Venkatram Vishwanath}
\author[evl]{Andrew Johnson}
\author[evl]{Jason Leigh}
\author[alcf,niu]{Michael E. Papka}

\address[evl]{Electronic Visualization Laboratory (EVL), University of Illinois at Chicago, 842 W Taylor St, Chicago, IL 60607, USA}
\address[mcs]{Mathematics and Computer Science, Argonne National Laboratory, 9700 S Cass Ave, Lemont, IL 60439, USA}
\address[alcf]{Argonne Leadership Computing Facility, Argonne National Laboratory, 9700 S Cass Ave, Lemont, IL 60439, IL, USA}
\address[niu]{Northern Illinois University, 300 Normal Road, DeKalb, IL 60115, USA}

\begin{abstract}

In-situ analysis has been proposed as a promising solution to glean faster insight and to reduce the amount of data written out to storage. A critical challenge here is that the reduced dataset needed to visualize a specific region of interest as the simulation is running is typically held on a subset of the nodes and needs to be written out to storage. Coupled multiphysics simulations also produce a sparse data pattern wherein data movement occurs among a subset of nodes. We evaluate the performance of these data patterns and propose several mechanisms for improving performance. Our mechanisms introduce intermediate nodes to implement multiple paths to transfer data on top of default routing algorithms and utilize topology-aware data aggregation to avoid shared bottleneck links. The efficacy of our solutions is evaluated through microbenchmarks and application benchmarks on an IBM Blue Gene/Q system scaling up to 131,072 compute cores. The results show that our algorithms achieve up to a 2X improvement in achievable throughput compared to the default mechanisms.

\end{abstract}

\begin{keyword}
multiple paths \sep sparse data movement \sep topology-aware aggregation \sep data-intensive \sep Blue Gene/Q
%% keywords here, in the form: keyword \sep keyword

%% PACS codes here, in the form: \PACS code \sep code

%% MSC codes here, in the form: \MSC code \sep code
%% or \MSC[2008] code \sep code (2000 is the default)

\end{keyword}

\end{frontmatter}

%% \linenumbers

\section{Introduction}
\label{sec:intro}
Simulation time in a supercomputer depends partially in achievable networking performance in the supercomputer. The achievable networking performance in its turn depends on the exact combination of communication patterns of applications and routing algorithms used by the supercomputer. For each communication pattern there exists routing algorithms resulting in high networking performance. However, while the communication patterns have wide spectrum and vary time to time, the routing algorithms have a limited variation and usually are optimized for typical communication patters. This results in high networking performance for favored communication patterns but low networking performance for other communication patterns. Low networking performance subsequentially leads to long simulation time. Reducing the communication time for these non-favored communication pattern reduces the running time of an application. Thus, improving the networking performance for these communication patterns are important in redecing simulation time. 

Most of the routing algorithms are designed to perform well for certain communication patterns. However, for other communication patterns, they do not perform as well usually due to the unblanced on physically links caused by changing in the communication patterns. Rebalancing loads on the physical links in these situations can lead to better achievable performance. 
In this work, we propose a set of approaches that improve network performance for applications in supercomputers by rebalancing the load on physical links. Our approaches include using both heuristic algorithms and formal models with mathematical solvers to search for paths to move data from a set of sources to a set of destinations. The data then be divided into smaller messages and put into queues to move along found paths. The actual data movement can be done by using any available libraries for communication on the supercompter. We realize our approaches in a framework called OPTIQ. Our framework allows to easily expand the our work by adding algorithms for search paths, different way to schedule data transfer and different way to transfer data. It also allow to extend the framwork to other systems.

Our contributions include:
\begin{itemize}
\item Bring an insight understanding of how networking happens in the Blue Gene/Q supercomputer. By experiments, we show patterns that current routing algorithms favor and what it doesn not and explain why it happens it that way.
\item Propose a set of approaches to improve networking throughput by taking advantage of unused links or balancing network on links. We also explain when each approach should be used ot result highest possible throughputs.
\end{itemize}

Our paper includes the follows. In the next section, we present previous works in optimizing or improving data movement in various supercomputers or computing systems. In section \ref{sec:system}, we give a brief introduction to Mira - a Blue Gene Q supercomputer that we used to run our experiments. Section \ref{sec:approach} is the main section, where we explain about our framework and details of our approaches for each component of the framework. We demonstrate the efficacy of our approaches via a set of benchmarks in section \ref{sec:benchmark}. In section \ref{sec:conclusion}, we conclude our work and give some ideas about future work.


\section{Related work}
\label{sec:relatedwork}

Improving data movement performance by balancing network load has been studied in previous works. In \cite{Valiant81:Routing}, Valiant proposed a randomized routing mechanism mathematically proved to be able to route data globally with no sharing links for what network. However the routing mechanism did not work well in local routing \cite{singh2003:GOAL} . \cite{Pifarre91} proposed minimal routing optimization. However, it did not work well with global optimization \cite{singh2003:GOAL} . To address the limitation of both approaches, \cite{singh2003:GOAL} proposed GOAL – globally oblivious adaptive locally to balance load using adaptive routing algorithm for torus networks.

GOAL paper \cite{singh2003:GOAL}
Several bgq paper

\cite{Rodriguez09} proposed oblivious routing schemes in extended generalized fat tree networks. It extended 2 algorithms called S-mod-k and D-mod-k to provide a better oblivious solution for slimmed networks. 

\cite{Prisacari13a} showed that there is potential to optimize all-to-all collective exchange communication pattern at system level in fat tree networks and proved it mathematically and via simulations. The work is then extended in \cite{Prisacari13b} to propose a generic method to determine optimal pattern-specific routing for eXtended Generalized Fat Tree (XGFT). The method used a hybrid combination of integer linear programming (ILP) and dynamic programming. Interconnection network is divided into small subdomains and ILP is used for each subdomain. The local solution is then combined using dynamic programming. The proposed method takes up to several hours for several thousands of compute nodes interconnect network.

In other supercomputers, \cite{garcia2013:CrayDragonfly} proposed two deadlock-free routing mechanisms that support o
n-the-fly adaptive routing on Cray XC30 system for Dragonfly networks.

In Blue Gene Q, \cite{Chen:BGQ} proposed a heuristic routing for Blue Gene/Q supercomputing systems. The routing path is computed dynamically at the routing time based on coordinates of source and destination, partition shape and message size. The systems route a packet along the longest dimension of a partition first, shortest last. BG/Q systems route a message using single path. Different messages however might be routed using different paths.

Some optimizations are done based on system’s specific characteristics such as \cite{Kumar:Allreduce} proposed optimization for MPI\_Allreduce in BG/Q based on the observation on number links and other hardware supports on compute nodes.

Our previous works in \cite{Vishwanath:GLEAN} and \cite{SDAV:Bui2014b} addressed a problem of data aggregation for I/O purposes very dense in I/O case. In these works, we proposed . In a more recent work \cite{hbui:bgq} we presented our approaches for improving the performance very sparse. In this paper, we give solution for medium dense communication patterns between M and N.

In the next section, we briefly introduce Mira, a Blue Gene Q supercomputer that we used in our experiments in demonstrating the efficacy of our approaches.


\subsection{Experiment system}
\label{sec:system}

In this section, we describe Mira - an IBM Glue Gene Q supercomputer in which we developed our algorithms and a framework for network load balancing and conducted our experiments. Mira \cite{Chen:BGQ}, with 48 compute racks (48K nodes and 768K cores) at the ALCF, provides 10 PFlops theoretical peak performance. Each node has a 16-core processor and 16 GB of memory.

The interprocess communications of Blue Gene/Q travel on a 5D torus network both for point-to-point and for collective communications. This 5D torus interconnects a compute node with its 10 neighbors at 2 GB/s theoretical peak over each link in each direction, making a total of 40 GB/s bandwidth in both directions for a single compute node. Because of packet and protocol overheads, however, only up to 90\% of the raw data rate (1.8 GB/s) is available for user data. The machine can be partitioned into non-overlapping rectangular submachines; these submachines do not interfere with each other except for I/O nodes and the corresponding storage system.

For interconnect network traffic, BG/Q supports both deterministic and dynamic routing \cite{Chen:BGQ}. In deterministic routing, packets are routed based on dimension-ordered routing, from the longest first to the shortest last. In dynamic routing, routing is still dimension-ordered, but it is programmable, enabling different routing algorithms to be used. Given the size of a certain message, routing is always the same, and its path is known before it is routed. These are the default routing algorithms and cannot be changed during run time. However, one can set which routing zone id to use by using the PAMI\_ROUTING environment variable. Since BG/Q uses single-path data routing, for sending/receiving a message only one link of the ten available is used. The details of routing can be found in \cite{Chen:BGQ}.

PAMI is a low-level communication library for BG/Q \cite{PAMI:Kumar}. PAMI provides low-overhead communication by using various techniques such as accelerating communication using threads, scalable atomic primitives, and lockless algorithms to increase the messaging rate. Since MPI is implemented on top of PAMI, direct use of PAMI would provide higher messaging rates as well as lower latencies in comparison with MPI.




\section{Approaches}
\label{sec:approach}

In the Blue Gene Q supercomputers, data transfers are routed along using default routing algorithm. The default routing algorithm is proved to be well-balanced in many communication patterns \cite{Chen:BGQ}. However, for certain communication patterns, it results in poor performance due to unbalanced networking load on physical links. In this section, we present four approaches aiming for balancing load on physical links. We also give a brief description of a framework called OPTIQ, a realizationf of our approaches.

\subsection{Path searching approaches}
Four approaches that we present in this section include two heuristic algorithms, one multiple paths data movement taking advantages of existing shortest paths algorithms and one model-based approach. All paths searching algorihtms here are centric algorithms i.e. run at every node in exact order. Hence, every node has the same results. The approaches need some information in advance such as size, topology, torus of the partition, coordinates of all nodes in the partition. The information is collected once at the begining. The algorithms also need to have tuples of data movement requests: source, destination and amount of data to be transferred from the source to the destination.

\subsubsection{Heuristic 1}
In this approach, we search for paths between each pair of source and destination. From a source, we explore from all the destination in all possible directions. Whenever we reach a destination, we mark the destination as found to no longer search for it on other explorering paths from the same source. In this algorithm, we do not limit exploring paths by any constraints. The algorithm is described in \textbf{Algorithm \ref{alg:h1}}.

\begin{algorithm}
\textbf{Input:} Set of pairs of source-destination (\textit{s$_i$, d$_i$}). Number of nodes \textit{n}. Graph of nodes. \\
\textbf{Output:} Set of paths: one path for a pair of source-destination \\
\\
Structures:
\begin{algorithmic}
\State struct arc \{int u, int v\};
\State struct path \{set of arcs\};
\end{algorithmic}
Init:
    \begin{algorithmic}
	\State queue$<$struct path$>$ \textit{exploring\_paths} = $\varnothing$;
	\State queue$<$struct path$>$ \textit{complete\_paths} = $\varnothing$;
	\State bool \textit{visisted}[\textit{n}][\textit{n}];
	\For {0 $<=$ {\it i}, \textit{j} $<$ \textit{n}} 
	    \State \textit{visited}[{\it i}][{\it j}] = false;
	\EndFor
    \end{algorithmic}
Main:

\begin{algorithmic}
    \Function {Heuristic\_search\_I}{}

    \While {exist a source \textit{s$_i$} with unvisted neighbor \textit{u}}
	\State check\_and\_add\_new\_path({\it s}$_i$, \textit{u}, null);
	\State Pick next \textit{s$_i$} in the sources
    \EndWhile

    \While {(\textit{exploring\_path} != $\varnothing$)}
	\State path \textit{p} = \textit{exploring\_paths}.pop();
	\State {\it u} = last vertex in the path {\it p};
	\For {each neighbor \textit{v} of \textit{u}}
	    \State check\_and\_add\_new\_path({\it u}, \textit{v}, {\it p});
	\EndFor
    \EndWhile

    \EndFunction
\\
    \Function{check\_and\_add\_new\_path}{int \textit{u}, int \textit{v}, path {\it op}}
	\If {(!{\it visited}[{\it u}][{\it v}])}
	    \State create a path \textit{np} = {\it op}
	    \State add arc $<$\textit{u}, \textit{v}$>$ to \textit{np}
	    \State enqueue \textit{np} to \textit{exploring\_paths}
	    \If {{\it v} is one of the destinations of \textit{s$_i$} of \textit{np}}
	        \State enqueue \textit{np} to \textit{complete\_paths}
            \EndIf
	    \State {\it visited}[\textit{u}][\textit{v}] = true;
	\EndIf
    \EndFunction
\end{algorithmic}

\caption{Heuristic Alg 1: Exploring all paths without constraints}
\label{alg:h1}

\end{algorithm}

The \textbf{Algorithm \ref{alg:h1}} can be divided into 2 parts. In the first part, which is in the first \textbf{while} loop of the function Heuristic\_search\_I, we start at every source and add 1-hop paths to \textit{exploring\_paths} queue. Those paths are the paths from sources to their neighbors. We need a \textit{break} statement after each adding to make sure that every source can a path before they can all start again. This is to help ...

In the second part, which is the second \textbf{while} loop of the function Heuristic\_search\_I, we pop an exploring path \textit{p}from the \textit{exploring\_paths}. From the last added vertex \textit{u} of \textit{p}, we exploring all edges from it to it neighbors and add new path \textit{np} = \textit{p} + newly explored edge. If any of its neighbors is final destination of source \textit{s$_i$}, we then add \textit{np} into \textit{complete\_paths}. We continue the work until all the paths are explored.

Time complexity: The graph has V vertices and E edges. We have K pairs of (source, destinatinon), each source has at most D neighborhoods, then the time complexity of the \textbf{Algorithm \ref{alg:h1}} is $O$(K * (|V| + |E|)). The time complexity is breakdown as following:
\begin{itemize}
\item First part: we have K pairs hence K sources, for each source we discover its D neighbors, thus time will be $O$(K*D).
\item Second part: For this part, we get a path out of \textit{exploring\_paths}, create new paths by exploring its neighbors that are not visited by its source and add the new paths back to \textit{exploring\_paths}. For each sources, every vertex and every edge can be visited in the worst case, the time complexity would be $O$(|V| + |E|) minus to the vertices and edges visited by the first part. As we have K sources, the time complexity is $O$(K * (|V| + |E|)).
\end{itemize}

\subsubsection{Heuristic 2}
In the Algorithm \ref{alg:h1}, we explore all paths without concerning the number of hops that a path would take or maximum number of paths using a physical link. This leads to a long path or many paths sharing a physical link. A longer path results in more intermediate nodes, hence higher total transfer time. Higher maximum number of paths sharing a link leads to less bandwidth for each path. Both of them can lead to degraded performance.

This algorithm is an extension from Heuristic 1, in which we limit the exploring paths by both the number of hops and minimizing the maximum load. We do so by maintaining a min heap of paths. We use the heap to contain all the exploring paths instead of using a queue to manage the exploring paths as in Heuristic 1. This leads to increasing algorithm's time complexity in heap extracting (as taking data from a queue is O(1), from a min heap is O(log(n)) with n entries in the heap), but increasing the quality of explored paths. In this approach, we have finer results as we explore from least congested paths. The comparison function of the heap is based on both the number of hops and maximum load on each path. 

As most of the work in Heuristic 2 is similar to Heuristic 1, we do not repeat them, but only describe the differences between the two. The differences include:
\begin{itemize}
\item Use min heap of paths instead of queue of paths.
\item Need to heapify everytime we insert/delete from the min queue.
\end{itemize}

In the following, we describe the comparison function, which the core of the min heap for heapification. We also describe how to calculate the max load from the current load when adding a new path. We compare 2 paths based on its current max load and hop length. Among the two, max load has higher priority as max load affects the performance more.

\begin{algorithm}

Heap element comparison:
    \begin{algorithmic}
	\Function{heap\_compare} {path p1, path p2, int maxload, int maxhops}
	    \If {both paths has max load greater than maxload} 
		\State choose the one with smaller number of hops.
	    \EndIf
	    \If {One path has max load greater but one path has max load smaller than maxload}
		\State Choose the one with smaller load.
	    \EndIf
        \EndFunction
    \end{algorithmic}

\caption{Heuristic Alg 2: Exploring all paths with hops length and max load constraints}
\label{alg:h2}
\end{algorithm}

\subsubsection{Heuristic 3}

\begin{algorithm}
\textbf{Input:} Set of pairs of source-destination (\textit{s$_i$, d$_i$}). Number of nodes \textit{n}. Graph of nodes. Number of shortest path \textit{k}\\
\textbf{Output:} Set of paths: \textit{k} paths for a pair of source-destination\\
Init:
    \begin{algorithmic}
        \State queue$<$struct path$>$ \textit{complete\_paths};
    \end{algorithmic}
Main:
\begin{algorithmic}
    \Function {Heuristic\_search\_II}{}
	\For {each pair of source-dest (\textit{s$_i$}, \textit{s$_i$})}
	    \While{less than k paths discovered || still have paths to discover}
		\State Use Yen's algorithm to search for the shortest path \textit{p}.
		\State Check if adding \textit{p} make the current load over max load.
		\State If not, add \textit{p} into \textit{complete\_paths}
	    \EndWhile
	\EndFor
    \EndFunction
\end{algorithmic}

\caption{Heuristic Alg 3: k shortest paths}
\label{alg:h3}

\end{algorithm}

In the \textbf{Algorithm \ref{alg:h3}}, we use Yen's algorithm to search for k shortest paths between \textit{s$_i$, d$_i$}.

\begin{comment}

\subsubsection{Job-based AMPL model}
In this section we present a formal model written in A Modeling Language for Mathematical Programming (AMPL). The model captures demands of data movement between a set of sources and a set of destinations. The model also capture the graph of physical links of a partition given for the application. The model is presented in Model 1.

We presented the model in AMPL to solve the problem.

\begin{verbatim}
set Nodes;
set Arcs within Nodes cross Nodes;
set Jobs;

param Delay {Arcs} default 0;
param Capacity {Arcs} >= 0 default Infinity;
param Source {Jobs};
param Destination {Jobs} default 0;
param Demand {Jobs} default 0;

var Flow {Jobs, Arcs} >= 0;
var Z >= 0;

var total_flow{(i,j) in Arcs} = 
sum {job in Jobs} Flow[job,i,j];

maximize obj: Z;

subject to

zero_flow {job in Jobs, i in Nodes}:
sum{(i,j) in Arcs} Flow[job,i,j] - 
sum{(j,i) in Arcs} Flow[job,j,i] = 
if (i == Source[job])  then Demand[job]*Z 
else if (i == Destination[job]) then -Demand[job]*Z 
else 0;

capacity {(i,j) in Arcs}:
total_flow[i,j] <= Capacity[i,j];
\end{verbatim}

Model explanation:
\begin{itemize}
\item sets: we have 3 sets: \textit{Nodes}, \textit{Arcs} and \textit{Jobs}. \textit{JobID}: is the set of transfers from sources to destinations. Each job is represented by a tuple (id, source, destination, demand (total data size to transfer)).
\item params: {\it Delay}: delay on each arc; {\it Capacity}: capacity of each arc; {\it Source}: set of sources; {\it Destination}: set of destinations; {\it Demand}: amount of data to be transferred in each job.
\item vars: \textit{Flow}: total flow of each job on each arc; \textit{Z}: is reversed of total time; \textit{total\_flow} total flow of all jobs going through an arc.
\item objective function: we want to minimize the time or maximize its reversed value i.e. maximize \textit{Z}.
\item constraints(subject to): \textit{zero\_flow}: total flow through a source is total going out of that source, total flow going through a destination is total flow going in that destination, for other nodes that total is 0; \textit{capacity}: total flow on an arc is less than its capacity.
\end{itemize}

\end{comment}

\subsubsection{Path-based model}

\begingroup
\fontsize{9pt}{10pt}\selectfont

\begin{verbatim}
set Nodes;
set Arcs within Nodes cross Nodes;
# Origin/Destination pairs
set OD within Nodes cross Nodes;
set Paths{OD};
set Path_Arcs{od in OD, p in Paths[od]}
    within Arcs;


param Capacity {Arcs} >= 0 default Infinity;
param Demand {OD} default 0;

var Flow {Paths} >= 0;
var Z >= 0;

var total_flow{(i,j) in Arcs} = 
    sum {p in Paths} Flow[job,i,j];

maximize obj: Z;

subject to

zero_flow {p in Paths, i in Nodes}:
sum{(i,j) in Arcs} Flow[job,i,j] - 
    sum{(j,i) in Arcs} Flow[job,j,i] =
    if (i == Source[job]) then Demand[job]*Z else 
    if (i == Destination[job]) then -Demand[job]*Z 
    else 0;

capacity {(i,j) in Arcs}:
total_flow[i,j] <= Capacity[i,j];
\end{verbatim}

\endgroup

Model explanation:
\begin{itemize}
\item sets: we have 5 sets: \textit{Nodes}, \textit{Arcs}, \textit{OD}, \textit{Paths} and \textit{Path\_Arcs}. \textit{JobID}: is the set of transfers from sources to destinations. Each job is represented by a tuple (id, source, destination, demand (total data size to transfer)).
\item params: {\it Capacity}: capacity of each arc; {\it Demand}: amount of data to be transferred in each job between a pair of orgin and destination.
\item vars: \textit{Flow}: total flow of each job on each arc; \textit{Z}: is reversed of total time; \textit{total\_flow} total flow of all jobs going through an arc.
\item objective function: we want to minimize the time or maximize its reversed value i.e. maximize \textit{Z}.
\item constraints(subject to): \textit{zero\_flow}: total flow through a source is total going out of that source, total flow going through a destination is total flow going in that destination, for other nodes that total is 0; \textit{capacity}: total flow on an arc is less than its capacity.
\end{itemize}

So far, we have presented different algorithms/approaches: here is when to use what:

\begin{table}[h]
\begin{center}
    \begin{tabular}{ | p{1.6cm} | l | p{3cm} |}
    \hline
    Approach & Time complexity & When to use \\ \hline
    Heuristic 1 & $O$(K * (|V| + |E|)) &  Used as first step for Heuristic 2\\ \hline
    Heuristic 2 & $O$(K * (|V| + |E|)) &  Very dense communication\\ \hline
    Heuristic 3 & $O$(K * (|V| + |E|)) & Sparse comminication \\ \hline
    Path-based model & $O$(K * (|V| + |E|)) & Medium dense where proportional throughtput can be gained \\
    \hline
    \end{tabular}

    \caption{Approaches: time complexity and usage}
    \label{tbl:experiment}

\end{center}
\end{table}

We realize algoirthms and other work in a framework named OPTIQ. The framework provides ways to improve data movement performance on the Blue Gene/Q supercomputer. Our framework does so by balancing loads on physical links on the Blue Gene/Q supercomputer.

\subsection{Framework}

Our framework has 3 main components: Path searching algorithms, Schedule and Transport and an extra component depicted in Figure \ref{fig:framework}.

\begin{figure}[!htb]
\vspace{-0.1in}
\centering
\includegraphics[scale=0.7]{figures/framework.pdf}
\vspace{-0.2in}
\caption{Three components of OPTIQ framework}
\vspace{-0.1in}
\label{fig:framework}
\end{figure}

The functionarity of each component is as following:
\begin{itemize}
\item Path searching: search for path to transfer data from a set of sources to a set of destination. Multiple or single paths can be found using a set of algorithm. User can decide what algorithm to be used or let the framework use a default algorithm.
\item Schedule: Split a buffer data that needed to tranfer into smaller messages and put those messages into a queue of transport layer to be transferred. It also handles incoming messages for itself and for forwardig them to its neighbors on a way to a message final destination.
As we route data in our own ways, we search for the paths, and we also need to schedule messages transfer. It includes sending local messages, forwarding messages form other sources, receiving data as the intermediate node or the destination node.

Order of messages into sending queue: 3 types of messages: local messages (needed to send), fowarding messages (needed to send), its receiving messages. first come first serve, local messages first. forwa

When there are multiple ranks per node, which one will be choosen to receive data at the next dest (forwarding). Single rank to do or many rank to do, currently every rank executes data transfer.

\item Transport: actually transfer an amount of data from one point to another point in the system.
\item Extra component: To get system specific information such as partition size, topology, coordinates, torus, and to compute neighbors of available nodes given to an application. Topology reading, coord, neighbors, torus, size, routing order, graph generated. Also set of benchmarks, tests.

the framework allow various options that allow to do: forwarding/local message sending order, algorithm selection, chunk size, transport implementations...
\end{itemize}


\section{Microbenchmarks}
\label{sec:microbenchmarks}
In this section, we show the efficacy of our solutions through a set of microbenchmarks for 2 cases: data movement for data coupling nodes and data movement between compute nodes and I/O.

\subsection{Efficacy of proxies and pipeline technique}

%On BG/Q, we tested the system with 16 cores per node all communicate with 16 cores on another node, they transferred in all same links, resulting in 1.7GB/s at most. Due to the symmetry of mapping, it's more likely to all cores belonging to a nodes to connect with all cores belonging another nodes than connect with cores at different nodes. With 20 links per nodes, using multiple paths for data movement on BG/Q can results in significant throughput improvement. 

In this microbenchmark, we transfer data between two nodes through an intermediate nodes using pipelining and compare results with transferring data without using pipeline and with a direct transfer (default) scheme. We varied the data size from 1K  to 128MB.

\begin{figure}[!htb]
\centering
\includegraphics[scale=0.3]{figures/pipeline_mpi.pdf}
\caption{Using pipelining technique to mitigate the waiting time at proxy nodes}
\label{fig:pipeline_mpi}
\end{figure}

Fig. \ref{fig:pipeline_mpi} shows that transferring data through a proxy without using pipelining results a 50\% hit in performance over a direct transfer. This is because the proxy needs to wait until entire data is ready before forwarding it to the destination. By using pipelining, for message less than 64KB,  pipelining does not improve performance, however, for a message size larger than 64KB, pipelining demonstrates an improve performance. At a  message size of 1MB, 2MB, and 4MB, we achieve 70\%, 80\%, and  90\% respectively of the direct transfer bandwidth, and with larger size messages, we achieved a similar performance. Thus, with large size messages, pipelining technique can be used to transfer data through proxies. However, with small size messages, the performance gained is insignificant. Much of the performance overhead is due to the underlying rendezvous protocol design in MPI on the BG/Q. Next, we use PAMI to improve the performance for small messages. The next subsection, we show the efficacy of using PAMI on transferring small size messages.

\subsection{Quantifying computation used for data movement at proxies}
In this section, we quantify the time spent by the CPU for data movement, and we show that it is so small that it has no effect on the total time.

\subsection{Data movement for data coupling nodes}
In this benchmark, we show feasibility of the approach using proxies to increase transfer throughput between 2 compute nodes. We choose the first and the last node of a partition of 128 compute nodes with 2x2x4x4x2 torus. As the partition is large enough we are able to choose 4 proxies to transfer data in 4 directions +B, +C, +D, +E . In each node, only one MPI rank is used (when we use multiple MPI ranks per node to send data to the same destination, they all take the same output link, thus using one MPI rank is still valid and making the experiment easier). The data is transferred in increasing size from 1KB up to 128MB of data, with the size doubled each time. We use MPI\_Put to transfer data from source node to proxy nodes and then from proxy nodes to destination node. Each transfer is repeated multiple times with different data to eliminate any cache effect and achieve stable performance. The average transfer throughput between 2 nodes is reported in Figure \ref{fig:4proxies}.

\begin{figure}[!htb]
\vspace{-0.1in}
\centering
\includegraphics[scale=0.3]{figures/4proxies}
\vspace{-0.2in}
\caption{Using 4 proxies to improve data transfer throughput between 2 nodes}
\vspace{-0.1in}
\label{fig:4proxies}
\end{figure}

As the figure shows, for the small messages, direct transfer yields better performance. With large message, proxy-based transfers outruns direct transfer with $2\times$ better performance. This is foreseen with the reasons we mentioned before: with small messages, extra time caused by injecting and copying messages is significantly larger than the transferring time. It happens in the opposite way with large messages. The message size threshold to switch from direction transfer to proxy-based transfer is 256KB, yielding 1.4GB/s per link. After the threshold, direct transfer slowly reaches to maximum ~1.6BG/s while proxy-based transfer continue to thrive until ~3.2GB/s. Thus, the benchmark shows that proxy-based approach is feasible and can result significant improvement.

As data movement in multiphysics applications is done by more than 2 single nodes, the second benchmark elucidates the feasibility and achievable throughput for data movement between two groups of nodes. In this experiment, we transfer data between two groups of nodes, wherein each group has 256 nodes in a 4x4x4x16x2 torus of a 2K nodes partition. One group is at one corner of the partition, the other one is at the other end of the partition. The data size is also from 1KB to 128MB with doubled size each step. The experiment is repeated for a number of times. We are able to choose 3 proxies for each node. The Figure \ref{fig:3groxies} shows the average throughput measured.

\begin{figure}[!htb]
\vspace{-0.1in}
\centering
\includegraphics[scale=0.3]{figures/3groxies}
\vspace{-0.2in}
\caption{Using 3 group of proxies to improve data transfer bandwidth between 2 group of nodes}
\vspace{-0.1in}
\label{fig:3groxies}
\end{figure}

In the figure, we once again see that with small messages, direct transfer is better than proxy-based transfer. The threshold for this case increases to 512KB. At that message size, direct transfer also reaches to its maximum throughput, while the proxy-based transfer still has big room to increase up to 2.4GB/s. The performance increases $1.5\times$ as predicted since 3 proxies are used for each nodes. This benchmark shows that proxy-based data transfer is feasible for data transfer between groups of nodes. And we can achieve significant improvement in certain cases.

As we have mentioned in Section \ref{sec:approaches}, we need at least k $>$ 2 proxies per each data transfer to benefit from proxies. The more proxies we can use the better performance we can gain. However, as the size of communicating groups increases, the number of proxies we can set up decreases. If we add more proxies beyond the maximum possible proxies, data movements by extra proxies intervene existing ones and eventually degrade overall performance. The Figure \ref{fig:num_groxies} demonstrates the situation.

\begin{figure}[!htb]
\vspace{-0.1in}
\centering
\includegraphics[scale=0.3]{figures/num_groxies}
\vspace{-0.2in}
\caption{Performance variance with number of proxies}
\label{fig:num_groxies}
\vspace{-0.1in}
\end{figure}

With the torus 4x4x4x4x2 and 2 groups of 32 nodes each, we can set up at most 4 groups of proxies along A+, A-, B+, B- dimensions. The 5th proxy is the source node itself. As Figure \ref{fig:num_groxies} shows, when we increases the number groups of proxies from 2 to 3 and to 4, the throughput increases from no-improvement, to 1.5$\times$ and to 2$\times$. Thus, with large message sizes, we can gain k/2 times performance with k being the number of proxies. However, when we increase number of proxies to 5, the performance starts to drop due to intervention among concurrent data movements. Therefore, choosing number of proxies together with their locations is important to maximize throughput.

The above three benchmarks demonstrate the efficaty of our solutions for data movement between compute nodes. In the next subsection, we evaluate our approaches in the case of data movement between compute nodes and I/O nodes. 

%\subsection{Sparse data patterns}


%In the next subsection, we present our benchmarks on the two data patterns generated.
%\subsection{Staging}

\subsection{Data Movement to I/O nodes}
We perform a weak scaling study with two sparse data patterns, and scale the number of cores from 2,048 to 131,072 cores on the Mira BG/Q system. 

\begin{itemize}
\item Pattern 1: Uniform distribution data where data size of a MPI rank is uniformly distributed between 0 and 8MB.  Data is generated by using \textit{srand()} and \textit{rand()} functions in C/C++ and using \textit{time(NULL)} as a seed.  Total data size is about 50\% of the dense data. The distribution of the data size is shown in Figure \ref{fig:uniform}.
\item Pattern 2: Pareto distribution data where many of MPI ranks have data size of 0 bytes or very low size, and a few of MPI ranks have data size of 8MB or close by. The total data size is about 20\% of the dense pattern. The distribution of the data is shown in Figure \ref{fig:pareto}
\end{itemize}

In the data pattern 1, data sizes are uniformly distributed among nodes. This data pattern can be seen when we want to analyze data from different regions with different resolutions. Depending on the resolution, data sizes may vary accordingly.

\begin{figure}[!htb]
\vspace{-0.2in}
\centering
\includegraphics[scale=0.3]{figures/uniform.pdf}
\caption{Pattern 1: Histogram of data sizes for 1,024 processes using time(NULL) function with size from 0 to 8MB}
\label{fig:uniform}
\vspace{-0.1in}
\end{figure}

On the other hand, the data pattern 2 represents the case where data are sparse but not uniformly distributed. There are many nodes with almost no data while some nodes have large volume of data. This data pattern happens where we want to write out data from a region of contiguous MPI ranks while ignoring other regions.

\begin{figure}[!htb]
\vspace{-0.1in}
\centering
\includegraphics[scale=0.3]{figures/pareto.pdf}
\vspace{-0.1in}
\caption{Pattern 2: Histogram of data sizes of 1,024 processes using Pareto distribution function with size from 0B to 8MB}
\label{fig:pareto}
\vspace{-0.1in}
\end{figure}

On data pattern 1, we write roughly 8GB at 2,048 cores and 274GB of data at 131,072 cores. On data pattern 2, we write 3.4GB at 2,048 cores to 119GB of data at 131,072 cores. We compare the performance of performing aggregation for 2 data patterns using our approach and default MPI Collective I/O.

\begin{figure}[!htb]
\vspace{-0.1in}
\centering
\includegraphics[scale=0.3]{figures/mira_agg.pdf}
\vspace{-0.2in}
\caption{Aggregation throughputs on Mira}
\vspace{-0.2in}
\label{fig:mira_agg}
\end{figure}

Figure \ref{fig:mira_agg} depicts the performance of our topology-aware multipath data movement approach in comparison to the default MPI-I/O for the two sparse data patterns as we scale from 2,048 cores to 131,072 cores on the Mira BG/Q system. On the data pattern 1 (uniformly distributed data), we observe  $2\times$ improvement at 2,048 cores. The performance increases as we scale and we achieve up to $3\times$ at 131,072 cores. On the data pattern 2 (pareto distributed data), we gain $1.5\times$ improvement at 2,048 cores and $2\times$ improvement at 131,072 cores. Thus, we observe that leveraging network interconnect topology and multipaths plays an important role at small scale and is critical at large scale. With the increased use of in-situ analysis for supercomputing,  sparse data patterns for I/O are becoming increasingly important and our approaches help provide more insights for improved performance.

\section{Application I/O benchmark}
\label{sec:app_benchmarks}
In this section, we demonstrate our solution on HACC I/O representing data movement between compute nodes and I/O nodes.

\subsection {HACC I/O}
HACC (Hardware/Hybrid Accelerated Cosmology Code) \cite{Habib:HACC} is a large-scale cosmology code suite that simulates the evolution of the universe through the first 13 billion years after the Big Bang. The simulation tracks the movement of trillions of particles as they collide and interact with each other, forming structure that transform into galaxies. During the runtime, HACC writes data periodically to storage system. The data can also be transferred from Mira to Tukey for data analysis and visualization. In both ways, data needs to go from compute nodes to I/O nodes first. In this benchmark, we use HACC I/O, an I/O benchmark written to evaluate performance of the I/O system for HACC, to show the data transfer performance from compute nodes to I/O nodes by writing to /dev/null. We compare the throughput of our mechanism to default MPI collective write on HACC I/O.
%\subsection{Staging}
%Presenting staging data from Mira to vis cluster Tukey. Analyze the results.

\subsection{Transferring data to I/O nodes}
In this experiments, we scale our experiments from 8,192 up to 131,072 compute cores to simulate the collision of $768^3$ to $2,816^3$ particles. We write only 10\% of the generated data with the amount of 2GB to 85GB of data. The data is written from processes with MPI ranks within the range [4*num\_processes/10, 5*num\_processes/10] with the num\_processes being the total number of MPI ranks in our application. We collect the bandwidth information and report the average. The results are shown in Figure \ref{fig:hacc_agg}

\begin{figure}[!htb]
\vspace{-0.1in}
\centering
\includegraphics[scale=0.3]{figures/hacc_agg.pdf}
\vspace{-0.1in}
\caption{Write throughput of HACC application to I/O nodes /dev/null}
\vspace{-0.1in}
\label{fig:hacc_agg}
\end{figure}

The results show that in both cases, the number of I/O nodes employed is more than default I/O nodes. However, in our case, the position and location of aggregators are chosen dynamically and are distributed uniformly brought in better performance. Overall we can get up to 50\% throughput improvement. Thus, dynamic selection of number of and location of aggregators based on size of data and interconnect topology is of paramount importance for sparse data movement.

\subsection{Energy consumption comparison}
In these micro-benchmarks, we show that when using multi-path data movement for sparse data patterns, we not only increase bandwidth but also save energy consumption by the system.

\begin{figure}[!htb]
\vspace{-0.1in}
\centering
\includegraphics[scale=0.3]{figures/power_cat.pdf}
\vspace{-0.1in}
\caption{Energy consumption by category at a node}
\vspace{-0.1in}
\label{fig:hacc_agg}
\end{figure}

\begin{figure}[!htb]
\vspace{-0.1in}
\centering
\includegraphics[scale=0.3]{figures/power_ncp.pdf}
\vspace{-0.1in}
\caption{Total energy consumption at scale}
\vspace{-0.1in}
\label{fig:hacc_agg}
\end{figure}


\section{Conclusion}
\label{sec:conclusion}

In this paper we propsed two approaches for balancing load on the Blue Gene/Q supercomputer using multipaths for data movement. We realized our approaches into a framework and demonstrated the efficacy of our works through a set of benchmarks. Overall, we improved the throughput by 40\% to more than 60\% on average for 3 different patterns in 91 experiments in up to 4096-node partitions. The performance however, can be improved up to 4X. Our work shows that to improve data movement performance, we need to take advantage of both applications' data movement patterns and system routing algorithms. For future, we plan to study the approaches in the Cray XE6 supercomputer and tested in real applications.


\section*{Acknowledgments}

This research used resources of the Argonne Leadership Computing Facility (ALCF) at Argonne National Laboratory, and is supported by the Office of Science of the U.S. Department of Energy under contract DE-AC02-06CH11357. We thank the ALCF team for discussions and help related to this paper.

%% The Appendices part is started with the command \appendix;
%% appendix sections are then done as normal sections
%% \appendix

%% \section{}
%% \label{}

\section*{References}

%% If you have bibdatabase file and want bibtex to generate the
%% bibitems, please use
%%
\bibliographystyle{elsarticle-num} 
\bibliography{sigproc}

%% else use the following coding to input the bibitems directly in the
%% TeX file.

%\begin{thebibliography}{00}

%% \bibitem{label}
%% Text of bibliographic item

%% \bibitem{}

%\end{thebibliography}
\end{document}
\endinput
%%
%% End of file `elsarticle-template-num.tex'.
