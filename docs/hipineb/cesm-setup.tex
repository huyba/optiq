Our experiment was carried on a partition of 512 nodes with 4 MPI/PAMI ranks per node. The positions of each component, and number of pairs of communication by rank and by node are shown in Table \ref{table:cesm_pair}

\begin{table}[!htbp]
   \centering
    \begin{tabular}{| l | c | r | r |}
    \hline
     \multirow{2}{*}{Model} & \multirow{2}{*} {Location} & \multicolumn{2}{ c| }{Num. of pairs of communication with Coupler} \\ \cline{3-4}
     & & Between Ranks & Between Nodes \\ \hline
     ATM & 0 - 1791 & 5227 & 2911 \\ \hline
     LND & 0 - 515 & 10390 & 2911\\ \hline
     ICE & 516 - 1791 & 3018 & 765 \\ \hline
     OCN & 1792 - 2047 & 2001 & 502 \\ \hline
     CPL & 0 - 1791 & & \\ \hline
    \end{tabular}
    \caption{Locations and number of pairs of communication between models in CESM}
    \label{table:cesm_pair}
\end{table}

The Table \ref{table:cesm_pair} shows the ranges of ranks (start-end) that host the models and the coupler. It also shows the number of pairs of communication between the models and the coupler.
