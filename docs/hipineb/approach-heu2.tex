\subsection{Heuristic Approach}
\label{sec:heuristic}

In this section we describe our heuristic to select paths for each source-destination pair, given the set of $k$ shortest paths $k\_paths$. In this approach we assume pairs of sources-destinations have different amounts of data called $demand$. Data of a pair is split into smaller chunks and assigned into its set of paths in such a way that minimize the maximum total amount of data assigned on physical links. At each time of assignment, we only assign an amount of $chunk$. This allows us to iterate through all pairs and assign $chunk$ into their paths. After assigning a $chunk$ to a path, all physical links comprising the path have an additional $chunk$ on its load.

Each path maintains a $maxload$ value, which is the maximum value of current loads on physical links comprising the path. After an assignment of a $chunk$ to a path, not only the path's $maxload$ value needs to be checked and updated but any paths that shares the path's links also needs to be checked and updated their $maxload$ values.

In order to reduce the total amount assigned on any physical link, we allow the pair with the largest amount of data select a path to assign $chunk$ amount data first. We maintain a max heap $heap$ of pairs with a pair with the largest remaining data amount being at the top of the heap. The assignment is as follows. We pop the pair at the top, select in its $k\_paths$ a path with minimum $maxload$, assign a $chunk$ of data to the path and update its remaining data. If there is still data to be assigned, we push it back to the $heap$ and do heapify. We update $maxload$ value on paths and loads on physical links. Our assigning finishes when all pairs assign all of their data to their paths. The pseudocode of our approach is presented in Algorithm~\ref{alg:heu}. 

\begin{algorithm}%[!htbp]
\SetAlgoLined\DontPrintSemicolon
\SetKwFunction{FindPaths}{FindPaths}
\SetKwFunction{HeuristicSearch}{HeuristicSearch}
\KwIn{Set of source-destination pairs $\mathcal{P}$ = \{(S, D) $|$ S, D $\in~N$\} and their $k$ shortest paths {$k\_paths$}. Data size $chunk$ for each assignment of data to paths.}
\KwOut{Set of source-destination pairs $\mathcal{P}$$_{out}$ with assigned data for each path.}
\SetKwProg{myproc}{Procedure}{}{}
\myproc{\FindPaths{}}{
    Make heap $heapP$ from $\mathcal{P}$.

    \While {($heapP$ !=  $\phi$)}
    {
	Heapify the $heapP$ \;
	$pair$ = $heapP$.pop() \;
	Select a path $p$ in $pair$'s $k\_paths$ with minimum $maxload$ value \;
	Assign $chunk$ data to $p$ \;
	Update $maxload$ value of $p$ and any paths that use $p$'s physical's links and corresponding physical links \;
	$pair$.demand -= $chunk$ \;

	\uIf {pair.demand $>$ 0}
	{
	    $heapP$.pushback($pair$);
	}
	\Else
	{
	    $\mathcal{P}$$_{out}$.add($pair$) \;
	}
    }
}

\caption{Heuristic to search paths for each source-destination pair from $k$ shortest paths.}
\label{alg:heu}
\end{algorithm}

In the Algorithm \ref{alg:heu}, we pick the pair with largest amount of remaining data first. This allows its paths and corresponding physical links being selected first. The pairs with lower remaining of data can select paths later.  Thus, load is balanced between pairs with higher load and pairs with lower load. Among all paths of a pair, the path with least $maxload$ is selected first. Thus, the load is balanced among paths belonging to the same pair.

The data assignment in this heuristic approach is greedy and local optimization. In the next section, we present the second approach, in which we employ a mathematical model to optimize data movement by assign optimal amount of data to paths.
