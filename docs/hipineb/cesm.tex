\section{Community Earth System Model (CESM)}
\label{sec:cesm}

\subsection{Introduction}

Community Earth System Model (CESM) is a coupled global climate model simulating the earth system consisting ice, land, ocean, atmospheric and other components \cite{CESM:Hurrell}. In order to provide flexibility in developing, models in CESM do not communicate directly with each other but via a coupler. The Figure \ref{fig:mct} shows the communication between 4 models: Atmosphere (ATM), Ice (ICE), Land (LND) and Ocean (OCN) with the Coupler (CPL).

\begin{figure}[!htb]
\vspace{-0.15in}
\centering
\includegraphics[scale=0.5]{figures/mct.pdf}
\vspace{-0.15in}
\caption{Model communicate with each other via a coupler}
\vspace{-0.15in}
\label{fig:mct}
\end{figure}

In this paper, we demonstrate the efficacy of our framework via data movements between the models and the coupler extracted from a real run of CESM.

\subsection{Experiments and results}

Our experiment was carried on a partition of 512 nodes with 4 MPI/PAMI ranks per node. The positions of each component, and number of pairs of communication by rank and by node are shown in Table \ref{table:cesm_pair}

\begin{table}[!htbp]
   \centering
    \begin{tabular}{| l | c | r | r |}
    \hline
     \multirow{2}{*}{Model} & \multirow{2}{*} {Location} & \multicolumn{2}{ c| }{Num. of pairs of communication with Coupler} \\ \cline{3-4}
     & & Between Ranks & Between Nodes \\ \hline
     ATM & 0 - 1791 & 5227 & 2911 \\ \hline
     LND & 0 - 515 & 10390 & 2911\\ \hline
     ICE & 516 - 1791 & 3018 & 765 \\ \hline
     OCN & 1792 - 2047 & 2001 & 502 \\ \hline
     CPL & 0 - 1791 & & \\ \hline
    \end{tabular}
    \caption{Locations and number of pairs of communication between models in CESM}
    \label{table:cesm_pair}
\end{table}

The Table \ref{table:cesm_pair} shows the ranges of ranks (start-end) that host the models and the coupler. It also shows the number of pairs of communication between the models and the coupler.



The result of the expriment is shown in Table \ref{table:cesm_results}

\begin{table*}[!htbp]
   \centering
    \begin{tabular}{| l | l | r | r | p{0.5cm} | p{0.5cm} | p{0.5cm} | p{0.5cm} |p{0.5cm} | p{0.5cm} |p{0.5cm} | p{0.5cm} |p{0.5cm} | p{0.5cm} |}
    \hline
    \multirow{3}{*}{Coupling} & \multirow{3}{*}{Type} & \multirow{3}{1cm}{BW (GB/s)} & \multicolumn{3}{ c| }{Num. of Paths} & \multicolumn{2}{ c| }{Hopbytes} & \multicolumn{2}{ c| }{Num of copies}& \multicolumn{2}{ c| }{Num of paths} & \multicolumn{2}{ c| }{Total data} \\ \cline{4-6}
    & & & \multirow{2}{0.5cm}{Total Paths} & \multicolumn{2}{ c| }{Per Job} & \multicolumn{2}{ c| }{Per Path (MB)} & \multicolumn{2}{ c| }{Per Path}& \multicolumn{2}{ c| }{Per Link}& \multicolumn{2}{ c| }{Per Link (MB)} \\ \cline{5-14}
    & & & & {Max} & Avg & Max & Avg & Max & Avg & Max & Avg & Max & Avg\\ \hline
    \multirow{3}{*}{CPL-ATM} & OPT    & 352.24 & 3987 & 5 & 1.37 & 72 & 20.61 & 480 & 128.50 & 20 & 5.21 & 44.62 & 22.81 \\ \cline{2-14}
    & HEU & 327.66 & 15022 & 15 & 5.16 & 12.90 & 1.37 & 162 & 17.22 & 75 & 16.73 & 13.18 & 5.00 \\ \cline{2-14}
    & MPI    & 276.71 & 2911 & 1 & 1.00 & 0.14 & 0.07 &  &  & 18 & 3.71 &  & \\ \hline
    \multirow{3}{*}{CPL-LND} & OPT    & 343.5 & 4004 & 7 & 1.38 & 70 & 20.85 & 448 & 130.54 & 18 & 5.24 & 44.62 & 22.77 \\ \cline{2-14}
    & HEU &  332.40 & 15107 & 14 & 5.19 & 12.90 & 1.36 & 168 & 17.12 & 75 & 16.81 & 13.31 & 5.00 \\ \cline{2-14}
    & MPI    & 278.90 & 2911 & 1 & 1.00 & 0.14 & 0.07 & 0 & 0.00 & 18 & 3.71 & 0 & 0 \\ \hline
    \multirow{3}{*}{CPL-OCN} & OPT    & 135.16 & 987 & 5 & 1.97 & 864 & 252.51 & 6048 & 1626.19 & 11 & 2.47 & 253.62 & 120.46\\ \cline{2-14}
    & HEU &  136.06 & 5924 & 35 & 11.80 & 157.06 & 9.98 & 2148 & 126.98 & 78 & 9.47 & 64.75 & 19.1346 \\ \cline{2-14}
    & MPI    & 104.44 & 502 & 1 & 1.00 & 0.14 & 0.07 & 0 & 0.00 & 9 & 3.12 & 0 & 0 \\ \hline
    \end{tabular}
    \caption{Throughput, total num of paths, number of paths per job, maximum and average values of hopbytes, number of copies, number of paths per link and amount of data per link for 3 couplings in 512 nodes (4 ranks/node) experiments.}
    \label{table:cesm_results}
\end{table*}

