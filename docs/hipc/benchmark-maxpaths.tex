\subsubsection{Number of paths input to model}

For the Optimization approach we need to input $k$ shortest paths for the solvers to search for an assignment of flow values on the $k$ paths. In this experiment we show the relationship between the number of paths input to the model, the corresponding data transfer throughput and the elapsed time for the AMPL model and solvers. We carried out the experiment in a 2048-node partition for all three patterns with source to destination ratio of 1:8, where 128 nodes communicate with 1024 nodes. We used 1 MPI/PAMI rank per node and 8 MB per communication. We varied the number of paths fed into the solvers from 4 to 16, 32 and 50. The performance is shown in Table \ref{table:pathsintomodel}. 
\begin{table}[h] %[!htbp]
   \centering
    \begin{tabular}{| l | p{0.5cm} | p{0.5cm} | p{0.5cm} | p{0.5cm} | p{0.75cm} |}
    \hline
     \multirow{2}{*}{Patterns} & \multirow{2}{*}{MPI} & \multicolumn{4}{ c| }{Number of paths} \\ \cline{3-6}
     & & 4 & 16 & 32 & 50 \\ \hline
     Disjoint & 61 & 29 & 84 & 104 & 197 \\ \hline
     Overlap & 59 & 82 & 192 & 224 & 308 \\ \hline
     Subset & 111 & 99 & 163 & 168 & 172 \\ \hline
    \end{tabular}
    \caption{\small Throughput (GB/s) with different number of paths input to the solvers.}
    \vspace{-0.15in}
    \label{table:pathsintomodel}
\end{table}
%As shown in the Figure \ref{table:pathsintomodel}, 
As we increase the number of paths, the performance improves. This is because with more paths the solvers have a larger search space, and thus can produce more optimal results to be used for data transfer. However, with increasing number of paths, we also increase the time for AMPL model to prepare and solvers to search for flow values for paths as shown in Table \ref{table:solvetime}.
\begin{table}[!htbp]
   \centering
   \begin{tabular}{| p {0.75cm}| p{0.5cm} | r | p{0.5cm} | p{0.5cm} | r | r | r | r |}
    \hline
    \multirow{2}{*}{Pattern} & \multicolumn{4}{ c| }{AMPL time (s)} & \multicolumn{4}{ c| }{Solve time (s)} \\ \cline{2-9}
    & 4 & 16 & 32 & 50 & 4 & 16 & 32 & 50 \\ \hline
    Disjoint & 13.9 & 187.7 & 123.0 & 224.0 & 0.06 & 6.6 & 4.4 & 84.0 \\ \hline
    Overlap & 13.6 & 51.9 & 134.6 & 198.7 & 0.09 & 16.6 & 179.4 & 530.3 \\ \hline
    Subset & 14.4 & 50.6 & 134.9 & 217.3 & 0.85 & 111.3 & 173.2 & 939.6 \\ \hline
    \end{tabular}
    \caption{\small AMPL and solving time.}
    \vspace{-0.15in}
    \label{table:solvetime}
\end{table}

