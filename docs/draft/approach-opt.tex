%\subsection{Model Optimization Approach}
\subsection{Optimization-based Approach}
\label{sec:optimization}

In this approach, we find the optimal assignment of data transfer along multiple paths from source nodes to destination nodes. Given the amount of data to be transferred, and the $k$ shortest paths from source to destination nodes (Algorithm~\ref{alg:h1}), we formulate the problem of finding paths for transferring data from sources to destinations as an optimization problem. The objective of the optimization problem is to minimize the total transfer time by finding paths that are uniformly loaded. Next, we describe the problem parameters.

The data transfer request of a source-destination pair is denoted as a $job$. The set of all jobs is denoted by $Jobs$. Each job has an amount of data to be transferred from its source to its destination. This amount of data for job $job$ is denoted by $Demand[job]$, which can be transferred over selected paths from a pre-computed set of paths $kpaths_{job}$ (computed prior in Algorithm~\ref{alg:h1}). A path is denoted by $p$. The amount of data transferred per unit time for job $job$ on path $p$ is called a flow of the job and is denoted as $flow(job, p)$. A path $p$ might comprise of one or more links. ($i$, $j$) denotes a link from vertex $i$ to vertex $j$. All links on path $p$ accommodate the same flow. $flow(job, p)_{ij}$ denotes the flow over link ($i$, $j$) of path $p$. $c(i,j)$ denotes the capacity of link ($i$, $j$). The link capacities are known link bandwidths of the interconnect. The objective of the optimization problem is to minimize time $t$ to transfer data for all jobs in $Jobs$ subject to two constraints. The decision variables are set of selected paths for a job and flow along each path $flow(job, p)_{ij}$. We describe our linear program formulation below.

Objective function:

\begin{center}
minimize $t$
\end{center}

Constraints:
\begin{itemize}

\item Total flow of a job is equal to the sum total of all its flows along its paths. For any $job$ in $Jobs$: 

\begin{equation} 
\label{eq:jobflow}
\sum_{\forall p \in kpaths_{job}} flow[job, p]  = \frac{Demand[job]}{t}
\end{equation}

\item Total flow of an arc is less than its capacity. For any arc ($i$, $j$):

\begin{equation} 
\label{eq:linkcapacity}
\sum_{\forall job \in Jobs}\sum_{\forall p \in kpaths_{job}} flow[job, p]_{ij} \leq c(i, j)
\end{equation}

\end{itemize}

%
The model is written in AMPL (A Mathematical Programming Language). The model is descrbied in \textbf{Model} 1.

\begin{algorithm}[!htbp]

\begingroup
\fontsize{9pt}{9pt}\selectfont

\begin{verbatim}

set Nodes;
set Arcs within Nodes cross Nodes;

set Jobs;
set Paths{Jobs};
set Path_Arcs{job in Jobs, p in Paths[job]} 
    within Arcs;

param Capacity{Arcs} >= 0 default Infinity;
param Demand {Jobs} default 0;

var Flow {job in Jobs, Paths[job]} >= 0;
var Z >= 0;

maximize obj: Z;

subject to

demand_con {job in Jobs}: sum {p in Paths[job]} 
	Flow[job,p] = Demand[job]*Z;

capacity_con {(i,j) in Arcs}:
  sum {job in Jobs, p in Paths[job]: 
    (i,j) in = Path_Arcs[job,p]} Flow[job,p] 
		<= Capacity[i,j];

\end{verbatim}

\endgroup

\caption*{\textbf{Model 1} Data movement optimization}
\label{mod:opt}

\end{algorithm}

The notions used in \textbf{Model} 1 are explained as follows:

\begin{itemize}
    \item sets: 
	\begin{itemize}
	    \item \textit{Nodes}: set of nodes in the network, each node represent a compute node in the supercomputer.
	    \item \textit{Arcs}: set of arcs in the network. Each arc represent a physical link in the supercomputer.
	    \item \textit{Jobs}: set of jobs. Each jobs has a source and a destination.
	    \item  \textit{Paths}: set of paths for each job.
	    \item \textit{Path\_Arcs}: set of arcs on each path of each job.
	\end{itemize}
    \item params: 
	\begin{itemize}
	    \item {\it Capacity}: capacity of each arc i.e. bandwidth of the physical link.
	    \item {\it Demand}: amount of data to be transferred of each job between a pair of source and destination.
	\end{itemize}
    \item vars:
	 \begin{itemize}
	    \item \textit{Flow}: flow of each job on a path. It can be seen as the proprotional bandwidth assigned for the job on that path.
	    \item \textit{Z}: is reversed of total time.
	\end{itemize}
    \item objective function: we want to minimize the time or maximize its reversed value i.e. maximize \textit{Z}.
    \item constraints(subject to): 
	\begin{itemize}
	    \item \textit{demand\_con}: flow of a job on equals to the demand of the job divided by the transfer time.
	    \item \textit{capacity\_con}: total flow on an arc is less than its capacity.
	\end{itemize}
\end{itemize}

The model takes a set of nodes, a set of arcs and their corresponding capacity, a set of jobs (source/destination pairs), a demand for each job, a set of paths for each job, and a set of arcs for each path as inputs. It searches for an assignment of flow values (proportional capacity) for paths of all the jobs such that the transfer time for demands of all jobs is minimum.

We feed the model into solvers together with data of nodes, arcs, capacity, paths for jobs and get the paths with given proportional bandwidth. Based on proportional bandwidth, each path can take proportional demand of a job.


The first constraint in Equation \ref{eq:jobflow} captures a job's flow equality. The total amount of data of a job $Demand[job]$ is transferred in a time $t$ through a set of paths in $kpaths_{job}$. For each path, the job $job$ is assigned a throughput $flow[job, p]$. The total data transferred for a job must be less than $Demand[job]$. Thus, the job's throughput (rate of data transfer) needs to be equal to the combined throughput on all its paths. The second constraint in Equation \ref{eq:linkcapacity} captures bound for a link's capacity. The total throughput of all jobs on a link should not exceed the link's capacity.

We used AMPL (A Modeling Language for Mathematical Programming) \cite{ampl-book} to model the optimization problem. We used the SNOPT solver to solve the linear program. The solution times are listed in Section \ref{sec:results}. In the next section, we present our experiments and results to demonstrate the efficacy of our approaches.

