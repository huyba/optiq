\subsection{Model Optimization Approach}
\label{sec:optimization}

In this approach, we are given a set of pairs of sources and destinations. For each pair there is an amount of data needed to transfer and a set of pre-computed k shortest paths that can be employed to transfer that data. We are also given the capacities of links that comprise the k shortest paths. We need to search for an assignment of capacities for the employed paths that results in the minimum transfer time for all pairs. 

Each data transfer request of a pair is denoted as a $job$. The set of all jobs is denoted as $Jobs$. Each job has an amount of data needed to transfer from its source to its destination. This amount of data is denoted as $Demand[job]$. In order to transfer the data a job can employ paths from a pre-computed set of paths (computed prior in \textbf{Algorithm \ref{alg:h1}}) denoted as $kpaths_{job}$. A path is denoted as $p$. An amount of capacity given for a job $job$ on a path $p$ is called a flow of the job and is denoted as $flow(job, p)$. A path $p$ might comprise from one to many links. All links ($i$, $j$) on path $p$ accomodate the same flow value for the job $flow(job, p)_{ij}$. $c(i,j)$ denotes the capacity of a link from vertex $i$ to vertex $j$.

The time to transfer data for all jobs in $Jobs$ is $t$. Our goal is to minize time $t$ to transfer data for all jobs in $Jobs$ while satisfying the two constraints:

Objective function:

\begin{center}
mininize $t$
\end{center}

Constrainsts:
\begin{itemize}

\item Total flow that a job needs equals the total flows given by its paths. For any $job$ in $Jobs$: 

\begin{equation} 
\label{eq:jobflow}
\sum_{\forall p \in kpaths_{job}} flow[job, p]  = \frac{Demand[job]}{t}
\end{equation}

\item Total flow of an arc is less than its capacity: For any arc ($i$, $j$):

\begin{equation} 
\label{eq:linkcapacity}
\sum_{\forall job \in Jobs}\sum_{\forall p \in kpaths_{job}} flow[job, p]_{ij} \leq c(i, j)
\end{equation}

\end{itemize}

%
The model is written in AMPL (A Mathematical Programming Language). The model is descrbied in \textbf{Model} 1.

\begin{algorithm}[!htbp]

\begingroup
\fontsize{9pt}{9pt}\selectfont

\begin{verbatim}

set Nodes;
set Arcs within Nodes cross Nodes;

set Jobs;
set Paths{Jobs};
set Path_Arcs{job in Jobs, p in Paths[job]} 
    within Arcs;

param Capacity{Arcs} >= 0 default Infinity;
param Demand {Jobs} default 0;

var Flow {job in Jobs, Paths[job]} >= 0;
var Z >= 0;

maximize obj: Z;

subject to

demand_con {job in Jobs}: sum {p in Paths[job]} 
	Flow[job,p] = Demand[job]*Z;

capacity_con {(i,j) in Arcs}:
  sum {job in Jobs, p in Paths[job]: 
    (i,j) in = Path_Arcs[job,p]} Flow[job,p] 
		<= Capacity[i,j];

\end{verbatim}

\endgroup

\caption*{\textbf{Model 1} Data movement optimization}
\label{mod:opt}

\end{algorithm}

The notions used in \textbf{Model} 1 are explained as follows:

\begin{itemize}
    \item sets: 
	\begin{itemize}
	    \item \textit{Nodes}: set of nodes in the network, each node represent a compute node in the supercomputer.
	    \item \textit{Arcs}: set of arcs in the network. Each arc represent a physical link in the supercomputer.
	    \item \textit{Jobs}: set of jobs. Each jobs has a source and a destination.
	    \item  \textit{Paths}: set of paths for each job.
	    \item \textit{Path\_Arcs}: set of arcs on each path of each job.
	\end{itemize}
    \item params: 
	\begin{itemize}
	    \item {\it Capacity}: capacity of each arc i.e. bandwidth of the physical link.
	    \item {\it Demand}: amount of data to be transferred of each job between a pair of source and destination.
	\end{itemize}
    \item vars:
	 \begin{itemize}
	    \item \textit{Flow}: flow of each job on a path. It can be seen as the proprotional bandwidth assigned for the job on that path.
	    \item \textit{Z}: is reversed of total time.
	\end{itemize}
    \item objective function: we want to minimize the time or maximize its reversed value i.e. maximize \textit{Z}.
    \item constraints(subject to): 
	\begin{itemize}
	    \item \textit{demand\_con}: flow of a job on equals to the demand of the job divided by the transfer time.
	    \item \textit{capacity\_con}: total flow on an arc is less than its capacity.
	\end{itemize}
\end{itemize}


The first constraint in (\ref{eq:jobflow}) captures job's flow equality. The total amount of data of a job $Demand[job]$ is transferred in a time $t$ through a set of paths in $kpaths_{job}$. For each path, the job $job$ is given a throughput $flow[job, p]$. Thus, the job's throughput needs to be equal to combined throughput provided by its paths. The second constraint in (\ref{eq:linkcapacity}) captures link's capacity inequality, in which total throughput of all jobs on a certain links does not execeed the links's capacity.

In the next section, we present our experiments and results to demonstrate the efficacy of our approaches.
