%\subsection{Communication Patterns}
%In this paper 
Our experiments covered three communication patterns: Disjoint, Overlap and Subset. For the commnication patterns, we demonstrated the efficacy of our algorithms in comparions with MPI\_Alltoallv.
We demonstrate the data movement performance of our OPTIQ framework and existing MPI's routines on the following communication patterns:
\begin{figure}[ht]
\vspace{-0.1in}
\centering
\includegraphics[scale=0.55]{figures/patterns.pdf}
\vspace{-0.1in}
\caption{Communication patterns}
\vspace{-0.1in}
\label{fig:patterns}
\end{figure}
\begin{itemize}
\item Disjoint sets: is many-to-many communication pattern where the sources and destinations are separated. It is a typical pattern in many applications.
\item Overlap: is a many-to-many pattern where sources and destinations are overlapped sets. CESM uses this communication patterns for its coupling communications.
\item Subset: is a many-to-many pattern where sources or destinations are a subset of the other. This pattern can be found CESM or in I/O aggregation.
\end{itemize}
We carried out a set of experiments to study the system's behavior in various patterns and to demontrate throughput improvement.
