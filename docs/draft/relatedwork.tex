\section{Related work}
\label{sec:relatedwork}

GOAL paper \cite{singh2003:GOAL}
Several bgq paper

Data movement optimization with specific network topology and communication matrix is equivalent to multi-commodity flow problem \cite{Leighton91}. For this problem, if integral flow is required, the problem modeled as integer linear programming (ILP) formulation is NP complete. However, if fractional flow is allowed, there exists linear programming (LP) formulation solvable in polynomial time. Many works later use the similar ILP or LP formulations when formulating.

\cite{Racke02} showed that it is possible to find an oblivious routing algorithm within a symmetric network such that the contention is only poly-logarithmic increment with the optimum of any traffic patterns. However, Raecke’s construction is not polynomial time. \cite{Azar03} proposed a polynomial time solution by developing linear programming formulation that can be solved with the Ellipsoid algorithm. However, the formulation grows exponentially with the network size. \cite{Applegate06} concluded that it is possible to obtain a robust routing that guarantees a nearly optimal utilization with a fairly limited knowledge of the applicable traffic demands.

\cite{Kinsy09} proposed an application-aware deadlock-free oblivious routing framework. The work presented a mixed integer-linear programming (MILP) approach together with a heuristic approach to produce deadlock-free application-aware routes that minimize the latency. The framework used heuristic algorithm to calibrate the MILP algorithm.

\cite{Rodriguez09} proposed oblivious routing schemes in extended generalized fat tree networks. It extended 2 algorithms called S-mod-k and D-mod-k to provide a better oblivious solution for slimmed networks. 

\cite{Prisacari13a} showed that there is potential to optimize all-to-all collective exchange communication pattern at system level in fat tree networks and proved it mathematically and via simulations. The work is then extended in \cite{Prisacari13b} to propose a generic method to determine optimal pattern-specific routing for eXtended Generalized Fat Tree (XGFT). The method used a hybrid combination of integer linear programming (ILP) and dynamic programming. Interconnection network is divided into small subdomains and ILP is used for each subdomain. The local solution is then combined using dynamic programming. The proposed method takes up to several hours for several thousands of compute nodes interconnect network.

\cite{Valiant81:Routing} proposed a randomized routing mechanism mathematically proved to be able to route data globally with no sharing links for what network. However the routing mechanism did not work well in local routing \cite{singh2003:GOAL} . \cite{Pifarre91} proposed minimal routing optimization. However, it did not work well with global optimization \cite{singh2003:GOAL} . To address the limitation of both approaches, \cite{singh2003:GOAL} proposed GOAL – globally oblivious adaptive locally to balance load using adaptive routing algorithm for torus networks. 

\cite{Chen:BGQ} proposed a heuristic routing for Blue Gene/Q supercomputing systems. The routing path is computed dynamically at the routing time based on coordinates of source and destination, partition shape and message size. The systems route a packet along the longest dimension of a partition first, shortest last. BG/Q systems route a message using single path. Different messages however might be routed using different paths.

\cite{garcia2013:CrayDragonfly} proposed two deadlock-free routing mechanisms that support on-the-fly adaptive routing on Cray XC30 system for Dragonfly networks.

At the middleware layer, communication libraries using low level libraries to provide common communication patterns for applications. Most common library specifications are MPI-2/3 and GASNet \cite{Yelick07:PGAS}. Some optimizations are done based on system’s specific characteristics such as \cite{Kumar:Allreduce} proposed optimization for MPI\_Allreduce in BG/Q based on the observation on number links and other hardware supports on compute nodes.

In our previous works in \cite{Vishwanath:GLEAN}, \cite{SDAV:Bui2014b}, \cite{hbui:bgq} we have already presented our approaches for improving the performance of data movement for certain cases very dense in I/O case, or very sparse in p2s2. In this paper, we give solution for medium dense communication patterns between M and N.
