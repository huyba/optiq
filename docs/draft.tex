\documentclass[letter]{article}

\usepackage{fullpage}
\usepackage{url}

\title{Draft}

\begin{document}
\maketitle

\section{Introduction}
Achievalbe performance depends on the combination of communication patterns and routing algorithm being used.

Load balancing, routing, application communication patterns.

In this work, we propose a set of approaches that improve network performance for applications in Blue Gene/Q supercomputers. 

\section{Related work}

\section{Experiment system}

\section{Approaches}

\subsection{Heuristic approaches}

\subsubsection{Heuristic 1}
In this approach, we search for paths to 

\subsubsection{Heuristic 2}

\subsection{Formal approaches}

\subsubsection{Job-based formulation}
Present the approach

Explain the approach

Limitation, move the second one.

\subsubsection{Path-based formulation}
Present the approach, explain and see how it fixes the issues.

\subsection{OPTIQ framework}

\section{Benchmarks}

\subsection{Experimental }
\subsubsection{Chunk size for pipeline}
\subsubsection{Various message sizes}
\subsubsection{Many ranks}
\subsubsection{Scaling}

\subsection{Communication patterns}
In this paper we demostrate data movement performance of our OPTIQ framework and existing MPI's routines on the following communication patters:

\begin{itemize}
\item All to many, many to all
\item I/O Aggregation: a special case of All to many (or many to many)
\item Many to many: disjoint/exclusive
\item Many to many: subset
\item Many to many: partially joint (not subset, not disjoint)
\item Many to many: sparse
\item Nearest neighbors
\end{itemize}

\subsection{All to many or Many to all}

\section{Conclusion}

\end{document}
