
\section{GLINT: using glint-example}
If finally you want to see what GLIMMER can do using the GLINT climate driver,
download the \texttt{glint-example} and try one of the provided example setups.
CD into the directory and try any of the config examples. Start glint\_example
by typing
\begin{verbatim}
    glint_example
\end{verbatim}
You will then be asked for a climate configuration file and an ice model 
configuration file.
For the climate file, a global example including precipitation and temperature timeseries
is provided. To let glint know about it, type
\begin{verbatim}
    glint_example.config
\end{verbatim}
For the ice model config, there are two examples, Greenland and North America. To chose either one, type
\begin{verbatim}
  gland20.config
\end{verbatim}
or
\begin{verbatim}
  namerica20.config
\end{verbatim}
respectively at the prompt asking for the config file, to start the model.
Both models are outputting three files each, containing different variables.
Every 100 years, a file \texttt{namerica20.hot.nc} or \texttt{gland20.hot.nc}, respectively is output, 
which can be used to hotstart the model later from any of the recorded stages.

As mentioned above, the model type (binary) to use can be stated in the configuration file,
or given using the \texttt{-m} option. Currently, the three model binaries that
come with GLIMMER are \texttt{simple\_glide, eis\_glide} and
\texttt{glint\_example}. The simple\_glide and eis\_glide drivers that are started using the glide\_launch.py Python script,
which needs to know which binary to address. The model binary can also be set as an environment variable \$GLIDE\_MODEL. 
However, as glint is called directly using the compiled binary \texttt{glint\_example} here, it is not necessary to further 
specify the model.
